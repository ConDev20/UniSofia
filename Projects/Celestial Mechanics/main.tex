\documentclass{article}

% Language setting
% Replace `english' with e.g. `spanish' to change the document language
\usepackage[bulgarian]{babel}
\usepackage{float}
% Set page size and margins
% Replace `letterpaper' with`a4paper' for UK/EU standard size
\usepackage[letterpaper,top=2cm,bottom=2cm,left=3cm,right=3cm,marginparwidth=1.75cm]{geometry}

% Useful packages
\usepackage{amsmath}
\usepackage{graphicx}
\usepackage[colorlinks=true, allcolors=blue]{hyperref}

\title{Проект по небесна механика}
\author{Константин Стефанов Константинов, 81993 Компютърни науки}
\date{05.07.2021}
\begin{document}
\maketitle

% \begin{abstract}
% Your abstract.
% \end{abstract}

\section{Пресметнете координатите и скоростите на планетите в деня, в който сте родени.}


\noindent Kакво ни е дадено:

\noindent Координатите на планета J от Слънчевата система се задават със следната формула: 
\begin{equation}
\begin{pmatrix}
x \\
y \\
z \\
\end{pmatrix} = 
\begin{pmatrix}
\cos(\theta) & -\sin(\theta) & 0 \\
\sin(\theta) & \cos(\theta) & 0 \\
0  & 0  & 1  \\
\end{pmatrix}
\begin{pmatrix}
1 & 0 & 0 \\
0 & \cos(i) & -\sin(i) \\
0  & \sin(i)  & \cos(i)  \\
\end{pmatrix}
\begin{pmatrix}
\cos(g) & -\sin(g) & 0 \\
\sin(g) & \cos(g) & 0 \\
0  & 0  & 1  \\
\end{pmatrix}
\begin{pmatrix}
a(\cos(u) - e)\\
a\sqrt{1-e^2} \sin(u)\\
0\\
\end{pmatrix}
\end{equation}
където (a, e, i, $t_0$, g, $ \theta $) са шестте орбитални елементи на планетата :

\begin{itemize}
   \item  a - дължина на перихелия
   \item  e - ексцентрицитет
   \item  i - наклоненост на планетата
   \item  $t_0$ - момент на преминаване през перихелия
   \item  g - аргумент на перихелия
   \item  $ \theta $ - дължина на възела
\end{itemize}
\noindent Данните за тези константи са взети от \href{https://ssd.jpl.nasa.gov/txt/aprx_pos_planets.pdf}{тук}.\\
$ \mu_j $ - относителната маса на дадена планета \\
u - ексцентрична аномалия \\
l - средна аномалия \\
$t_b$ - времето в години от 2000г. до рождената дата.\\

\noindent Изчисляваме g по формулата : $g = \Tilde{w} - \theta $, където $\Tilde{w}$ е $\Omega$ от таблицата.\\
В сила е уравнението на Кеплер: l = u - e.$sin{u}$\\
За намирането на l извършваме следните изчисления:\\
Намираме $l_{(2000)} = L_{2000} - g - \theta$ \\
l(средна аномалия) = $n . t_b + l_{(2000)} $  \\

\noindent От безкрайната рекурентна формула: $u= l + e.sin(l+e.sin(...))$ изчисляваме ексцентричната аномалия.
\begin{equation}
v = Q * \frac{an}{1- e.cos(u)}*
\begin{pmatrix}
a(\cos(u) - e)\\
a\sqrt{1-e^2} \sin(u)\\
0\\
\end{pmatrix}
\end{equation}, където Q е от основната формула на сферичната тригонометрия.\\

В следващата страница са представени таблица с резултатите от пресмятанията за координатите и скоростите на планетите, както и стойността на $\mu$ за тях.
\clearpage

\begin{center}
 \begin{tabular}{||c | c||} 
 \hline
 Планетa & $\mu$  \\ [0.5ex] 
 \hline\hline
 Меркурий & 1/6023600 \\ 
 \hline
 Венера & 1/408523  \\
 \hline
 Земя & 1/328900  \\
 \hline
 Марс & 1/3098708  \\
 \hline
 Юпитер & 1/1047,34 \\
 \hline
 Сатурн & 1/3497,8  \\
 \hline
 Уран & 1/22902,9  \\
 \hline
 Нептун & 1/19042  \\
 \hline
 Плутон & 1/135000000  \\[1ex]
 \hline
\end{tabular}
\end{center}

\begin{figure}[h]
\centering
\includegraphics[width=0.95\textwidth]{figures/plantes_data.png}
\end{figure}

\section{Пресметнете координатите и скоростите на планетите в деня, в който сте родени.}
Eлементите на Делоне (L, G, $\Theta$, l, g, $\theta$), където L, G и $\Theta$ се изразяват чрез орбиталните елементи : \\

\setlength{\tabcolsep}{30pt} % Default value: 6pt
\renewcommand{\arraystretch}{1.5} % Default value: 1

\begin{center}
\begin{tabular}{ c |  c  | c } 

 $L = \mu \sqrt{\gamma} \sqrt{a} $   & $L - G = \mu \sqrt{\gamma} \sqrt{a}(1-\sqrt{1-e^2})$ & $G - \Theta = \mu  \sqrt{\gamma}  \sqrt{a}  \sqrt{1-e^2} (1- cos(i)) $ \\ 

\end{tabular}
\end{center}

\noindentкъдето $\gamma = \sqrt{1+\mu}$ \\


\setlength{\tabcolsep}{6pt} % Default value: 6pt
\renewcommand{\arraystretch}{1} % Default value: 1
\noindent Елементите на Поанкaре от първи вид се изразяват чрез елементите на Делоне:\\
\begin{center}
\begin{tabular}{ c   c   c } 
    &(L, L-G, $G-\Theta$, $\lambda $, $\Tilde{w}$, $\Omega$)&
\end{tabular}
\end{center}
$\lambda = l + g + \theta$\\
$\Tilde{w} = g + \theta$\\
$\Omega = \theta$\\

\noindentЕлементите на Поанкaре от втори вид се изразяват чрез тези от първи:\\
\setlength{\tabcolsep}{6pt} % Default value: 6pt
\renewcommand{\arraystretch}{2} % Default value: 1
\begin{center}
\begin{tabular}{ c   c   c } 
    &$L=L$&\\
    &$\xi=\sqrt{2(L-G)}\cos{g+\theta}$&\\
    &$p=\sqrt{2(G-\Theta})\cos{\theta}$&\\
    &$\lambda = l+g+\theta$&\\
    &$\eta=-\sqrt{2(L-G})\sin{g+\theta}$&\\
    &$q=-\sqrt{2(G-\Theta})\sin{\theta}$&\\
\end{tabular}
\end{center}

\setlength{\parskip}{10em}
\noindentВ следващите страници са представени таблици с резултатите от пресмятанията на елементите на Делоне и Поанкаре от първи и втори вид за всички планети, както и код от програмата.\\

\noindent\textbf{Елементи на Делоне и Поанкре от първи и втори вид:}
\begin{figure}[h]
\centering
\includegraphics[width=1\textwidth]{figures/Picture1.png}
\end{figure}\\
\newpage

\noindent\textbf{Код от задача 1:}\\
\begin{figure}[h]
\centering
\includegraphics[width=1\textwidth]{figures/1.2.png}
\end{figure}

\newpage

\noindent\textbf{Код от задача 2:}\\
\begin{figure}[h]
\centering
\includegraphics[width=1\textwidth]{figures/2.1.png}
\end{figure}
 
\newpage
\noindent\textbf{Спомагателни функции за двете задачи:}\\
\begin{figure}[h]
\centering
\includegraphics[width=1\textwidth]{figures/4.png}
\end{figure}

 
\end{document}